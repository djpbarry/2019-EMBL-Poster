\block{\begin{blockbody}\section{What is bioimage analysis?}
			\tfont
            \begin{minipage}{0.5\textwidth}
                  Bioimage analysis refers to automated, quantitative analysis and is distinct from manual analysis, which could be considered one of two things:\par
                  \begin{itemize}
                    \item Purely qualitative assessment.
                    \item Manual quantitative analysis.
                  \end{itemize}\par
			\end{minipage} \hfill
			\begin{minipage}{0.47\textwidth}
                          \begin{tikzpicture}[every node/.style={inner sep=0}]
                \node[rblock] (exp) at (0.0, 0.0) {Experimental Outputs};
                \node[rblock] (example) at (2.5 * \dist, 0.0) {Representative Example};
                \path[rarrow] (exp) -- (example);
              \end{tikzpicture}\par
			\end{minipage}\par
            \vs
			\begin{minipage}{0.5\textwidth}
				Truly quantitative bioimage analysis refers to workflows that are...\par
                  \begin{itemize}
                    \item Automated so far as possible.
                    \item Unbiased so far as is possible.
                    \item Consistent across all images \& experiments.
                    \item Reproducible.
                  \end{itemize} \par
			\end{minipage} \hfill
			\begin{minipage}{0.47\textwidth}
                          \begin{tikzpicture}[every node/.style={inner sep=0}]
                \node[gblock] (exp) at (0.0, 0.0) {Experimental Outputs};
                \node[gblock] (data) at (-1.25*\dist, -1.3*\dist) {Quantitative Data};
                \node[gblock] (model) at (1.25* \dist, -1.3*\dist) {Modelling};
                \path[garrow] (exp) -- (data);
                \path[garrow] (data) -- (model);
                \path[garrow] (model) -- (exp);
              \end{tikzpicture}\par
			\end{minipage}\par
		\end{blockbody}}