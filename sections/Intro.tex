\block{\begin{blockbody}\section{Introduction}
    \tfont
    Current trends in cell biology research indicate an increasing exploitation of three-dimensional experimental model systems. While cutting-edge microscopy technologies permit the routine acquisition of three-dimensional datasets, there is currently a lack of user-friendly software to analyse such images, particularly in the open source domain. Exploiting the functionality of several existing open-source projects and implemented as a plugin for the popular FIJI platform, GIANI (https://djpbarry.github.io/Giani) is a new analysis tool designed to address this need. Open-source and fully automated with a user-friendly interface, GIANI is designed specifically with batch-processing large quantities of 3D image data in mind. The design primarily facilitates segmentation of nuclei and cells, followed by quantification of morphology and protein expression. However, GIANI also provides scripting functionality and command line tools, such that users can incorporate its functionality into their own scripts and macros and easily run GIANI on a HPC cluster. We demonstrate the utility of GIANI by quantifying the heterogeneous expression of GATA3 and YAP1 in treated and control mouse embryos.\par
\end{blockbody}}